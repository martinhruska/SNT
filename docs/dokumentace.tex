\documentclass[a4paper, 12pt]{article}
\usepackage[left=1.5cm, text={18cm, 25cm}, top=2.5cm]{geometry}
\usepackage[utf8]{inputenc}
\usepackage[czech]{babel}
\usepackage{cite}
\usepackage{graphicx}
\usepackage{float}
\usepackage{amsmath}
\usepackage{amssymb}
\usepackage{tikz}
\usepackage{url}
\usepackage{comment}
\newcommand{\myuv}[1]{\quotedblbase #1\textquotedblleft}
\newcommand{\rcps}[0]{Resource Constraint Project Scheduling}

\title{\rcps}
\author{Martin Hruška\\xhrusk16@stud.fit.vutbr.cz}

\date{}
\begin{document}

\maketitle

\section{Úvod}
\label{sec:intro}
% speek about np hardness of problem

\section{Definice a analýza \rcps}
Napřed formálně definujme \rcps. Ten lze formálně definovat šesticí $(V, p, E, R, B, b)$ \cite{artigues:2007}, kde
\begin{enumerate}
    \item $V=\{A_0, \ldots, A_{n+1}\}$ je množina aktivit, které mají být vykonány.
    Dle konvence je $A_0$ aktivita reprezentující začátek rozvrhu a $A_{n+1}$ aktivita reprezentující konec rozvrhu.
    \item $p: V~\leftarrow \mathbb{N}$ je funkce udávající dobu trvání aktivit.
    \item $E \subseteq V\times V$ je precedenční relace.
    Pokud $(A_i, A_j) \in E$, tak aktivita $A_i$ musí být provedena před aktivitou $A_j$.
    \item $R=\{R_1,\ldots,R_q\}$ je množina zdrojů.
    \item $B: R \leftarrow \mathbb{N}$ je funkce udávající kapacitu zdrojů.
    \item $b: V~\times R \leftarrow \mathbb{N}$ je funkce udavájící množství určitého zdroje, které vyžaduje daná aktivita pro
    svoje provádění.
\end{enumerate}

Řešením tohoto problému je vektor $<S_0,\ldots,S_{n+1}>$, kde $S_i$ je doba začátku aktivity $A_i$.
Toto řešení můžeme nazývat rozvrhem.
Řešení musí splňovat omezení daná precedenční relací a dostupností zdrojů, což lze formálně zapsat následovně:
\begin{equation}
 \forall (A_i,A_j) \in E: S_j - S_i \geq p(A_i)
\end{equation}
\begin{equation}
 \forall R_k \in R \forall t \geq 0: \sum_{A_i \in A_t} b(A_i, R_k) \leq B(k)
\end{equation}
kde $A_t=\{A_i \in A\,|\, S_i \leq t \leq S_i+p(A_i)\}$.

Kritériem optimálnosti u~\rcps je tzv. \emph{makespan}, což je doba začátku provádění aktivity $A_{n+1}$.
Optimalizačním problémem tedy je nalézení řešení \rcps\ úlohy s~minimální makespan.

V~dalším textu jsou uvažovány \emph{single-mode} \rcps\ úlohy.
Single-mode znamená, že aktivity dané úlohy mají pouze jeden mód činnosti,
narozdíl od tzv. \emph{multi-mode} úloh jejichž aktivity mají více módu činnosti
a každý mód má odlišné požadavky na spotřebu zdrojů.

%define goal
% solving methods of rcps
\subsection{PSPLib}
% what is it and how should I use it
\subsection{SAT Solving v~RCPS}
% define sat problem
% what is sat solving
% how is it used in my model
% used sat solvers

\section{Konceptuální model}
% picture with desing -- conceptual way
% formalizovat prevod zadani rcps modelu na sat formuli


\section{Architektura}
% describe classes, data structures
% what is sat solver, translater, parser

\section{Experimenty}
% this is clear

\section{Závěr}
% make this beatiful!

\newpage
\bibliography{literatura}
\bibliographystyle{czechiso}
\end{document}
