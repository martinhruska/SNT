\documentclass[a4paper, 12pt]{article}
\usepackage[left=1.5cm, text={18cm, 25cm}, top=2.5cm]{geometry}
\usepackage[utf8]{inputenc}
\usepackage[czech]{babel}
\usepackage{cite}
\usepackage{graphicx}
\usepackage{float}
\usepackage{amsmath}
\usepackage{amssymb}
\usepackage{tikz}
\usepackage{url}
\usepackage{comment}
\usepackage[longend,ruled,vlined,commentsnumbered,linesnumbered,czech]{algorithm2e}
\newcommand{\myuv}[1]{\quotedblbase #1\textquotedblleft}
\newcommand{\rcps}[0]{Resource Constraint Project Scheduling}

\title{\rcps}
\author{Martin Hruška\\xhrusk16@stud.fit.vutbr.cz}

\date{}
\begin{document}

\maketitle

\section{Úvod}
\label{sec:intro}
% speek about np hardness of problem

\section{Analýza \rcps}
V~této sekci bude napřed formálně definován \rcps\ problém a optimalizační problém související s~jeho řešení.
Dále bude popsána knihovna PSPLIB a její formát popisu instancí RCPS problému.
Následně budou zkoumány možnosti a způsoby využití SAT solving při řešení RCPS problému.

\subsection{Formální definice RCPS}
\rcps je definován šesticí $(V, p, E, R, B, b)$ \cite{artigues:2007}, kde
\begin{enumerate}
    \item $V=\{A_0, \ldots, A_{n+1}\}$ je množina aktivit, které mají být vykonány.
    Dle konvence je $A_0$ aktivita reprezentující začátek rozvrhu a $A_{n+1}$ aktivita reprezentující konec rozvrhu.
    \item $p: V~\leftarrow \mathbb{N}$ je funkce udávající dobu trvání aktivit.
    \item $E \subseteq V\times V$ je precedenční relace.
    Pokud $(A_i, A_j) \in E$, tak aktivita $A_i$ musí být provedena před aktivitou $A_j$.
    \item $R=\{R_1,\ldots,R_q\}$ je množina zdrojů.
    \item $B: R \leftarrow \mathbb{N}$ je funkce udávající kapacitu zdrojů.
    \item $b: V~\times R \leftarrow \mathbb{N}$ je funkce udavájící množství určitého zdroje, které vyžaduje daná aktivita pro
    svoje provádění.
\end{enumerate}

Řešením tohoto problému je vektor $<S_0,\ldots,S_{n+1}>$, kde $S_i$ je doba začátku aktivity $A_i$.
Toto řešení můžeme nazývat rozvrhem.
Řešení musí splňovat omezení daná precedenční relací a dostupností zdrojů, což lze formálně zapsat následovně:
\begin{equation}
 \forall (A_i,A_j) \in E: S_j - S_i \geq p(A_i)
\end{equation}
\begin{equation}
 \forall R_k \in R \forall t \geq 0: \sum_{A_i \in A_t} b(A_i, R_k) \leq B(k)
\end{equation}
kde $A_t=\{A_i \in A\,|\, S_i \leq t \leq S_i+p(A_i)\}$.

Kritériem optimálnosti u~\rcps je tzv. \emph{makespan}, což je doba začátku provádění aktivity $A_{n+1}$.
Optimalizačním problémem tedy je nalézení řešení \rcps\ úlohy s~minimální makespan.

V~dalším textu jsou uvažovány \emph{single-mode} \rcps\ úlohy.
Single-mode znamená, že aktivity dané úlohy mají pouze jeden mód činnosti,
narozdíl od tzv. \emph{multi-mode} úloh jejichž aktivity mají více módu činnosti
a každý mód má odlišné požadavky na spotřebu zdrojů.

% solving methods of rcps
\subsection{PSPLIB}
% what is it and how should I use it
%Dokumentovaný nástroj očekává jako vstup instanci RCSP problému zadanou ve formátu používaném knihovnou \emph{PSPLIB}, jež
%ude v~této sekci stejně jako formát vstupu stručně popsána.
PSPLIB je knihovna združíjící sady instancí \rcps\ problému v~jeho různých variantách a jejich optimální a heuristická řešení \cite{psplib}.
Knihovna používá ke generování instancí problému vlastní nástroj \emph{ProGen}.
Ve shodě se zadáním projektu byly použita data knihovny PSPLIB jako vstupy dokumentováného nástroje.
Použití knihovny je vhodné také díky možnosti porovnávat výstupy vytvořeného programu s~optimálními řešeními, které
knihovna pro mnoho instancí problému obsahuje.
Příklad vstupních dat lze najít v~adresáří \emph{input}.
TODO DECRIBE STRUCTURE OR DETELE THIS

\subsection{SAT Solvinig}
Metody a nástroje SAT solvingu se zabývají řešením SAT problému (který je NP-úplný),
který odpovídá na otázku, zdali je daná formule výrokové logiky splnitelná.
V~našem případě uvažujeme formuli výrokové logiky v~\emph{konjunktnivní} normální formě (CNF).
formule je definovaná přes množinu proměnných, které nabývají hodnoty z~množiny $\{true, false\}$
a je složena z~následujících prvků:
 \begin{itemize}
 	\item \emph{Literálu}, což jsou samotné proměnné nebo jejich negace
 	\item \emph{Klausulí}, což jsou disjunkce ($\vee$) literálů
	\item Samotná výroková formule v~CNF je pak konjunkcí ($\wedge$) klausulí.
 \end{itemize}.

Ohodnocení formule je konkrétní zobrazení jejich proměnných na hodnoty $\{true, false\}$.
Formule je \emph{splnitélná}, pokud existuje ohodnocení, při kterém je celá formule vyhodnocena
jako $true$.
Takové ohodnocení se potom nazývá modelem formule v~CNF.

Programy řešící SAT problém se nazývají \emph{SAT Solvery}.
V~poslední době proběhl v~oblastni SAT solverů výrazný pokrok a mnohé z~nich jsou již
schopny  řešit SAT problémy obsahující velké formule.
Porovnání možností a výkonu SAT solverů lze najít na \cite{www:sat}.

Základním algoritmem použivaným v~SAT Solvingu (a dnešních SAT Solverech) je
\emph{Davis–Putnam\-Logemann–Loveland (DPLL) algoritmus} \cite{dpll:1960, dpll:1962}.
Algoritmus lze najít v~\ref{alg:dpll} a bude následovat jeho krátký popis.

Funkce \emph{propagate} ohodnotí proměnné na základé aktuálního částečného ohodnocení tak, že
provedená nová ohodnocení jsou důsledkem aktuálního stavu.
Pokud při tomtu ohodnocení nevznikne konflikt, t.j. jedna klausule při učitém ohodnocení některé z~proměnných nabyde hodnoty $true$,
zatímco jiná (konfliktní) klausule se při takovémto ohodnocení stane $false$, tak pokud jsou již ohodnoceny všechny proměnné, je vrácena
hodnota $true$.
Pokud konflikt nenastane, ale nejsou ohodnoceny všechny proměnné tak, je zvýšena úroveň zanoření při rozhodování (proměnná decision\_level)
a je zavolána funkce \emph{decide}, která rozhodne, která další proměnná bude ohodnocena a na které se tedy bude další výpočet větvit.

Pokud funkce propagate vrátí konfliktní klausuli, tak je zavolána funkce \emph{analyze}, která určí ke které úrovni zanoření při ohodnocování
proměnných se bude navracet.
Všechna ohodnocení provedena hloubějí než v~této úrovni jsou pak zrušena.
V~moderních SAT Solverech funkce analyze také dokáže vytvořit \emph{naučenou} klausuli, která je důsledkem aktuální databáse klausulí a která
má popisovat příčinu posledního konfliktu konfliktu.

\begin{algorithm}
\label{alg:dpll}
    \KwIn{CNF formule}
    \KwOut{$true$ pokud je formule splnitelná, jinak $false$}
    decision\_level = 0\;
    \While{$true$}
    {
        propagate()\;
        \eIf{Není konflikt}
        {
            \eIf{Všechny proměnné přiřazeny}
            {
                \Return $true$\;
            }
            {
                ++decesion\_level\;
                decide()\;
            }
        }
        {
            analyze()\;
            \eIf{nalezen konflikt na nejvyšší úrovni}
            {
                \Return $false$\;
            }
            {
                backtrack()\;
            }
        }
    }
\caption{DPLL Algoritmus}
\end{algorithm}

V této práci byly použity SAT Solvery \emph{MiniSAT} \cite{www:minisat} a \emph{Glucose} \cite{www:glucose}, která
používají další optimalizace k zlepšení efektivity výpočtu.
MiniSAT byl vybrán na základě článku \cite{horbach:10}, kde byl úspěšně použit právě pro řešení RCPS problému.
Glucose je solver založený na MiniSATu, ale implementující některé další optimalizace, a proto s ní budou provedeny
experimenty, které si kladou za cíl zjistit, zdali přináší zlepšení pro výkonu při řešení RCPS problému.
% how is it used in my model
% used sat solvers

\section{Konceptuální model}
% todo jak se řeší sat problémem
V této sekci zabývající se konceptuálním bude popsána redukce RCPS problému na SAT problému.
Uvedená redukce vychází z \cite{horbach:10}.
\subsection{Redukce na SAT problém}
% formalizovat prevod zadani rcps modelu na sat formuli


\section{Architektura}
% picture with desing -- conceptual way
% describe classes, data structures
% what is sat solver, translater, parser

\section{Experimenty}
% this is clear

\section{Závěr}
% make this beatiful!

\newpage
\bibliography{literatura}
\bibliographystyle{czechiso}
\end{document}
